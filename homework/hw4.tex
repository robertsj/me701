\documentclass[11pt]{article}
\usepackage{mathtools,hyperref,booktabs,fullpage, txfonts}
\usepackage[amssymb,cdot]{SIunits}
\usepackage[utopia]{mathdesign}     

%\usepackage[table]{xcolor}
\usepackage{amsmath}
\usepackage{hyperref}
\usepackage{longtable}
\usepackage{fullpage}
 
\usepackage{listings}
\usepackage{color}
\definecolor{dkgreen}{rgb}{0,0.6,0}
\definecolor{gray}{rgb}{0.5,0.5,0.5}
\definecolor{mauve}{rgb}{0.58,0,0.82}
\definecolor{lightgray}{rgb}{0.9,0.9,0.9}
\lstset{language=Python,
  aboveskip=3mm,
  belowskip=3mm,
  showstringspaces=false,
  columns=flexible,
  basicstyle={\ttfamily},
  numbers=left,
  numberstyle=\small\color{gray},
  keywordstyle=\color{blue},
  commentstyle=\color{dkgreen},
  stringstyle=\color{mauve},
  breaklines=true,
  breakatwhitespace=true,
  tabsize=3
}

\definecolor{lightgray}{gray}{0.93}

\pagestyle{empty}
\setlength\parindent{0pt}
\renewcommand{\thefootnote}{\fnsymbol{footnote}}
 
\makeatletter
\renewcommand\section{\@startsection{section}{1}{\z@}%
                                  {-3.5ex \@plus -1ex \@minus -.2ex}%
                                  {2.3ex \@plus.2ex}%
                                  {\normalfont\bfseries}}
\makeatother


\begin{document}

{\large
  \begin{center}
    {\bf ME 701 -- Development of Computer Applications In Mechanical Engineering \\ 
         Homework 4 -- Due 10/6/2017}         
  \end{center}
}
 
Instructions:  Your solutions to the following should be contained in
one file named {\tt lastname\_firstname\_hw3.py} and uploaded to Canvas.


\section*{Problem 1 -- Basic Regular Expressions}

Define a regular expression that matches all of the items 
in the left column but none in the right column:
\begin{tabular}{ll}
 pit  & pt\\
 spot & Pot\\
 spate & peat\\
 slap two & part\\
 respite &
\end{tabular}

\section*{Problem 2 -- More Complex Regex}

Consider the file {\tt regex\_sample\_mcnp.txt} (in the examples folder).  There are two blocks of text in that file that look like
\begin{verbatim}
 1tally       18        nps =       10000
+                                     GOOD COUNTS                                                              
           tally type 8    pulse height distribution.                   units   number         
           particle(s): heavyions
 
 cell  10                                                                                                                              
      energy   
    0.0000E+00   0.00000E+00 0.0000
    3.0000E-01   0.00000E+00 0.0000
    2.0000E+02   3.35200E-01 0.0141
      total      3.35200E-01 0.0141
\end{verbatim}
Use regex to process the file and produce a dictionary of the form 
  \begin{lstlisting}
  d{18: {'energy': [0, 3e-1, 2e2], 'value': [0, 0, 3.352e-1], 'sigma'=[0.0, 0.0, 0.0141]},
    28: {... so on and so forth
  \end{lstlisting}
Here, 18 and 28 are integer identifiers for particular results of interest (called a ``tally'').  The energy, value, and uncertainty arrays are the corresponding numerical data.

\section*{Problem 3 -- Regression}

Go out and find some interesting data for which a suitable model has been (or could be) defined.  Use that data and the tools presented in class to determine the coefficients to that model that produce the (1) least-squares and (2) minimax fit.  \\

Note, you are all smart engineers, so I assume you can find interesting data.  If you can't, some possible options are
\begin{itemize}
 \item Getting atomic mass data and generating the coefficients to the semi-empirical mass formula.
 \item Acquiring (maybe even from an MNE faculty member) some measured thermal-hydraulic measurements used to define a correlation (like Dittus-Boelter, etc.)
\end{itemize}
These models can be linear or nonlinear, but they should not be trivial.




\section*{Problem 4 -- ODEs}

Use SciPy to solve the following:
\begin{enumerate}
 \item $y' = y + 1$ for $y(0) = 1$ and $t \in [0, 10]$.
 \item $y''' = y - y'$ for $y(0)=y'(0)=y''(0)=1$ and $t \in [0, 10]$.
 \item $y' = 1000 y + 1$ and $z' = 0.0001 z + 1000 y$ for $y(0) = z(0) = 1$ and $t \in [0, 10]$.
 \item $y' = y^2 + 1$ for $y(0) = 1$ and $t \in [0, 10]$.  How would you apply good-old Euler's method to this problem?  Do it!
 \item $-y'' + y = 1$ for $y(0) = y(10) = 1$.  This is a BVP!  Look up the ``shooting method'' or {\tt bvp\_solver}. 
\end{enumerate}


\end{document}
