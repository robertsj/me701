\documentclass[11pt]{article}
\usepackage{mathtools,hyperref,booktabs,fullpage}
%\usepackage[amssymb,cdot]{SIunits}
%\usepackage[utopia]{mathdesign}     
\usepackage{xcolor}
\usepackage{amsmath}
\usepackage{amssymb}
\usepackage{hyperref}
\usepackage{longtable}
\usepackage{fullpage}
\usepackage{enumitem}
\setlist{nolistsep}

%\definecolor{lightgray}{gray}{0.93}

\pagestyle{empty}
\setlength\parindent{0pt}
\renewcommand{\thefootnote}{\fnsymbol{footnote}}
 
\makeatletter
\renewcommand\section{\@startsection{section}{1}{\z@}%
                                  {-3.5ex \@plus -1ex \@minus -.2ex}%
                                  {2.3ex \@plus.2ex}%
                                  {\normalfont\bfseries}}
\makeatother


\begin{document}

{\large
  \begin{center}
    {\bf ME 701 -- Development of Computer Applications In Mechanical Engineering \\ 
         Homework 8 --Due 11/17/2017 \\
    }
  \end{center}

\setlength{\unitlength}{1in}

\begin{picture}(6,.1) 
\put(0,0) {\line(1,0){7.35}}         
\end{picture}
}

{\bf Instructions}:  All source code should be placed in a single 
TAR file of the form {\tt lastname\_firstname\_hw8.tar}.  All 
code should be compiled with a Makefile, as specified in Problem 3.

\section*{Problem 1 -- Arrays and Matrices}

Consider the (unimplemented) functions:
\begin{verbatim}
// C++
double interpolate(double x_new, double *x, double *y, int n, int order)
{
  /* ... */
}
! Fortran
double precision function interpolate(x_new, x, y, n, order)
    double precision, intent(in) :: x_new
    double precision, dimension(:), intent(in) :: x, y
    integer, intent(in) :: n, order
    ! ...
end function interpolate
\end{verbatim}
The functions should return the value of {\tt y\_new} at {\tt x\_new} 
interpolated from the points {\tt x} and {\tt y}.  The interpolation 
should be of the order given by {\tt order} (first is linear, second 
is quadratic, etc.). The number of points used is {\tt n}. 


Your tasks:
\begin{enumerate}
 \item Implement these functions in C++ and Fortran.
 \item Produce corresponding Python modules using SWIG and f2py.  
       Note, think carefully about the interpolate function and its 
       signature.
 \item Produce a Makefile that automates all of the compilation.
       Make sure that any system-dependent paths are {\it easily changed}
       by modifying a variable in your Makefile (e.g., for the Python
       path needed for SWIG).
\end{enumerate}






\section*{Problem 2 -- Monte Carlo}

Monte Carlo integration really shines for multidimensional problems, which is
why it works so well for integrating the 6-dimensional phase space of 
time-independent neutron transport problems.  In this problem, we'll use a 
very simple test case to illustrate the power of Monte Carlo integration and 
assess the performance of C++ and Fortran using a variety of 
compiler options.  For this problem, use the language you did not 
use for Problem 1. \\

{\bf Background}:  We've done numerical quadrature using pre-defined 
abscissa with corresponding weights.  Monte Carlo is good at solving integrals
by simulation.  For example, consider a unit square with a concentric circle that 
touches the center of each of the square's for sides.  If you threw 
100 darts, the number of darts inside the circle divided by 100 would 
be pretty close to $\pi/4$ (think about that if it's not obvious!).  
We can use this approach to solve a variety of integrals.  Consider any 
integrand of $n$ variables $f(\mathbf{x})$ where $\mathbf{x} \in \mathbb{R}^n$.
For simplicity, assume each component $x_i$ of $\mathbf{x}$ is 
restricted to $x_i \in [0, 1]$, and also assume that $f \in [0, 1]$.  
Then we proceed as follows:
\begin{enumerate}
 \item Generate random samples for each $x_i$, e.g., $x_i = \tt{rand}()$, where 
       $\tt{rand}$ is the random number generator in your language of choice.
       Call the resulting vector of samples $\mathbf{x}'$.
 \item Generate a random ``rejection'' sample $a = \tt{rand}$.
 \item If $a < f(\mathbf{x}')$, add 1 to the ``in the circle'' bin (i.e., it's 
       under the curve)
 \item Otherwise, do another sample until we've done as many as we want.
\end{enumerate}

Consider the integral
\begin{equation}
 I_n = \int^1_0 dx_1 \int^1_0 dx_2 \ldots \int^1_0 dx_n 
      (x_1 + x_2 + \ldots + x_n)^2 \, .
\end{equation}
We want to integrate this function  using 
\begin{enumerate}
 \item the midpoint rule (i.e., use the centers of multidimensional boxes)
 \item Monte Carlo
\end{enumerate}
It seems weird that the use of random numbers could ever be beneficial for 
this, but it turns out to be wickedly good for high-dimensional problems.

\vspace{11pt}
\textcolor{purple}{{\bf Deliverables}}: 
\begin{enumerate}
 \item Write a program in C++ {\bf or} Fortran to compute 
       the integral and the {\it relative, absolute error} (i.e.,
       $|I^{appx}-I^{ref}|/|I^{ref}|$
       with both 
       quadratures for arbitrary $n$ and an $m$, where 
       $m^n$ is the number of 
       of evaluations of the integrand.  (For the midpoint rule, 
       $m$ defines the number of divisions along a given axis.)
       I have provided template files for you to start with in 
       both languages.
       The main program should accept $n$ and $m$ as command line
       arguments.  It should output a file named {\tt output.txt} 
       that contains one 
       line of the following form: {\tt I I\_mp I\_mc}, where
       I, I\_mp, and I\_mc are the reference, midpoint, and Monte 
       Carlo values. 
 \item Plot the  {\it relative, absolute error} for
       $n = 1$, $3$, and $10$.  Use enough integrand evaluations 
       to obtain an absolute relative error of
       0.001 or less, and use a log-log scale.  You should use Python 
       for running your program, reading the output file, and 
       plotting the results. Name that file {\tt hw4p2.py}.  
 \item For the largest $n$ and $m$ values you needed, time your 
       Monte Carlo routine for the case of (1) no optimization and (2) 
       ``-O3'' optimization and report the result in your summary PDF file.
       This could take some time (for the unoptimized case), so 
       plan accordingly.
\end{enumerate}



\end{document}
